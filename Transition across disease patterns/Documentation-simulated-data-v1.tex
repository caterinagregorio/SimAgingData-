% Options for packages loaded elsewhere
\PassOptionsToPackage{unicode}{hyperref}
\PassOptionsToPackage{hyphens}{url}
%
\documentclass[
]{article}
\usepackage{amsmath,amssymb}
\usepackage{iftex}
\ifPDFTeX
  \usepackage[T1]{fontenc}
  \usepackage[utf8]{inputenc}
  \usepackage{textcomp} % provide euro and other symbols
\else % if luatex or xetex
  \usepackage{unicode-math} % this also loads fontspec
  \defaultfontfeatures{Scale=MatchLowercase}
  \defaultfontfeatures[\rmfamily]{Ligatures=TeX,Scale=1}
\fi
\usepackage{lmodern}
\ifPDFTeX\else
  % xetex/luatex font selection
\fi
% Use upquote if available, for straight quotes in verbatim environments
\IfFileExists{upquote.sty}{\usepackage{upquote}}{}
\IfFileExists{microtype.sty}{% use microtype if available
  \usepackage[]{microtype}
  \UseMicrotypeSet[protrusion]{basicmath} % disable protrusion for tt fonts
}{}
\makeatletter
\@ifundefined{KOMAClassName}{% if non-KOMA class
  \IfFileExists{parskip.sty}{%
    \usepackage{parskip}
  }{% else
    \setlength{\parindent}{0pt}
    \setlength{\parskip}{6pt plus 2pt minus 1pt}}
}{% if KOMA class
  \KOMAoptions{parskip=half}}
\makeatother
\usepackage{xcolor}
\usepackage[margin=1in]{geometry}
\usepackage{longtable,booktabs,array}
\usepackage{calc} % for calculating minipage widths
% Correct order of tables after \paragraph or \subparagraph
\usepackage{etoolbox}
\makeatletter
\patchcmd\longtable{\par}{\if@noskipsec\mbox{}\fi\par}{}{}
\makeatother
% Allow footnotes in longtable head/foot
\IfFileExists{footnotehyper.sty}{\usepackage{footnotehyper}}{\usepackage{footnote}}
\makesavenoteenv{longtable}
\usepackage{graphicx}
\makeatletter
\def\maxwidth{\ifdim\Gin@nat@width>\linewidth\linewidth\else\Gin@nat@width\fi}
\def\maxheight{\ifdim\Gin@nat@height>\textheight\textheight\else\Gin@nat@height\fi}
\makeatother
% Scale images if necessary, so that they will not overflow the page
% margins by default, and it is still possible to overwrite the defaults
% using explicit options in \includegraphics[width, height, ...]{}
\setkeys{Gin}{width=\maxwidth,height=\maxheight,keepaspectratio}
% Set default figure placement to htbp
\makeatletter
\def\fps@figure{htbp}
\makeatother
\setlength{\emergencystretch}{3em} % prevent overfull lines
\providecommand{\tightlist}{%
  \setlength{\itemsep}{0pt}\setlength{\parskip}{0pt}}
\setcounter{secnumdepth}{-\maxdimen} % remove section numbering
\ifLuaTeX
  \usepackage{selnolig}  % disable illegal ligatures
\fi
\IfFileExists{bookmark.sty}{\usepackage{bookmark}}{\usepackage{hyperref}}
\IfFileExists{xurl.sty}{\usepackage{xurl}}{} % add URL line breaks if available
\urlstyle{same}
\hypersetup{
  pdftitle={Documentation simulated data MM trajectories v1},
  pdfauthor={Caterina Gregorio, Valentina Manzoni},
  hidelinks,
  pdfcreator={LaTeX via pandoc}}

\title{Documentation simulated data MM trajectories v1}
\author{Caterina Gregorio, Valentina Manzoni}
\date{April 2025}

\begin{document}
\maketitle

\hypertarget{data-and-organization}{%
\section{Data and organization}\label{data-and-organization}}

\hypertarget{schematic-description-of-the-scenarios}{%
\subsection{Schematic description of the
scenarios}\label{schematic-description-of-the-scenarios}}

\begin{longtable}[]{@{}
  >{\centering\arraybackslash}p{(\columnwidth - 12\tabcolsep) * \real{0.1222}}
  >{\centering\arraybackslash}p{(\columnwidth - 12\tabcolsep) * \real{0.2667}}
  >{\centering\arraybackslash}p{(\columnwidth - 12\tabcolsep) * \real{0.1222}}
  >{\centering\arraybackslash}p{(\columnwidth - 12\tabcolsep) * \real{0.1222}}
  >{\centering\arraybackslash}p{(\columnwidth - 12\tabcolsep) * \real{0.1222}}
  >{\centering\arraybackslash}p{(\columnwidth - 12\tabcolsep) * \real{0.1222}}
  >{\centering\arraybackslash}p{(\columnwidth - 12\tabcolsep) * \real{0.1222}}@{}}
\toprule\noalign{}
\endhead
\bottomrule\noalign{}
\endlastfoot
\textbf{Scenario number} & \textbf{Folder name}* & \textbf{Number of
disease patterns} & \textbf{Number of individuals} & \textbf{Observation
scheme} & \textbf{Under reporting} & \textbf{Released} \\
1 & schema\_xa\_3000 & 2 & 3 000 & Regular, 2 years (A) & No &
\(\checkmark\) \\
2 & schema\_xb\_3000 & 2 & 3 000 & Semi-regular, 3/5 years (B) & No &
\(\checkmark\) \\
3 & schema\_xc\_3000 & 2 & 3 000 & Irregular (C) & No &
\(\checkmark\) \\
4 & schema\_xa\_under\_3000 & 2 & 3 000 & Regular, 2 years (A) & Yes &
\(\checkmark\) \\
5 & schema\_xb\_under\_3000 & 2 & 3 000 & Semi-regular, 3/5 years (B) &
Yes & \(\checkmark\) \\
6 & schema\_xc\_under\_3000 & 2 & 3 000 & Irregular (C) & Yes &
\(\checkmark\) \\
7 & schema\_xa\_1000 & 2 & 10 000 & Regular, 2 years (A) & No &
\(\checkmark\) \\
8 & schema\_xb\_1000 & 2 & 10 000 & Semi-regular, 3/5 years (B) & No &
\(\checkmark\) \\
9 & schema\_xc\_1000 & 2 & 10 000 & Irregular (C) & No &
\(\checkmark\) \\
10 & schema\_xa\_under\_1000 & 2 & 10 000 & Regular, 2 years (A) & Yes
& \\
11 & schema\_xb\_under\_1000 & 2 & 10 000 & Semi-regular, 3/5 years (B)
& Yes & \\
12 & schema\_xc\_under\_1000 & 2 & 10 000 & Irregular (C) & Yes & \\
\end{longtable}

*100 datasets corresponding to different pseudo-random generations of
the data are contained in each folder.

\hypertarget{dataset-structure}{%
\subsection{Dataset structure}\label{dataset-structure}}

The structure of the datasets are coherent across scenarios. Each
dataset is organized in a long-format i.e.~each row corresponds to a
different visit. Subjects can have different number of rows depending on
the scenario and time in the study. Below the variables contained in
each dataset are described:

\begin{itemize}
\item
  \texttt{dataset\_id} : identification number of the dataset (from 1 to
  100).
\item
  \texttt{subject\_id}: identification number of the subject (from 1 to
  the number of subjects according to the scenario.
\item
  \texttt{age\_baseline}: age at the entry of the study in years.
\item
  \texttt{age\_exit}: age at the end of the observation period (death,
  loss to follow up or administrative end of the study) in years.
\item
  \texttt{dth}: binary variable indicating whether the subject died (0:
  alive at the end of the observation period; 1 dead at the end of the
  observation period).
\item
  \texttt{time\_in\_study}: time in years from the study entry until the
  end of the observation (\texttt{age\_exit}-\texttt{age\_entry}) in
  years.
\item
  \texttt{cov1}, \texttt{cov2}, \texttt{cov3}: binaries variables
  indicating three different exposures that can be used as possible
  predictors of transitioning between the multimorbidity states and
  death. They are assumed to be fixed and measured at the study entry.
\item
  \texttt{visit\_number}: identification number for the visit.
\item
  \texttt{age}: age at the study visit in years.
\item
  \texttt{ndis}: number of prevalent diagnoses at the study visit.
\item
  from \texttt{anemia} to \texttt{peripheral\_vascular\_dis}: 60 binary
  variables indicating a specific diagnosis at the study visit. All
  diseases are assumed to be chronic and irreversible (they can't be
  ``turned off'').
\end{itemize}

\hypertarget{description-of-the-observation-schemes}{%
\subsection{Description of the observation
schemes}\label{description-of-the-observation-schemes}}

\begin{itemize}
\item
  \textbf{Schema A- Regular 2 years:} visits are every two years from
  baseline until death/ end of the observation due to
  loss-to-follow-up/administrative end of the study. The median number
  of visits per subject is 5 (IQR: 3-7).
\item
  \textbf{Schema B- Semi-Regular similar to SNAC-K:} visits are every 6
  years from baseline if the subject is 60-78 and every 3 years if they
  are 78+ until death/end of the observation due to
  loss-to-follow-up/administrative end of the study. The median number
  of visits per subject is 2 (IQR: 2-3).
\item
  \textbf{Schema C - Irregular:} visits are at irregular intervals and
  they are different among subjects. The median number of visits per
  subject is 6 (IQR: 3-10).

  \hypertarget{definition-of-under-reporting}{%
  \subsection{Definition of
  under-reporting}\label{definition-of-under-reporting}}

  Underreporting was defined as a random selection of diagnoses for five
  pre-specified diseases (Chronic Kidney Disease, Dementia,
  Deafness/Hearing Loss, Depression, and Osteoarthritis/Other
  Degenerative Joint Diseases) that were not detected during a visit.
  For simplicity, the probability of non-detection was kept constant
  across both time and diseases.
\end{itemize}

\hypertarget{data-generating-mechanism}{%
\section{Data generating Mechanism}\label{data-generating-mechanism}}

For each dataset \(n\) (\(n = 1,2, \dots, N\)) and for each subject
\(k\) (\(k =1,2,\dots, N_{\text{sim}}\)):

\hypertarget{population-composition}{%
\subsubsection{\texorpdfstring{\textbf{\emph{Population
composition}}}{Population composition}}\label{population-composition}}

\begin{enumerate}
\def\labelenumi{\arabic{enumi}.}
\item
  Draw \texttt{cov1} from a binomial distribution:\\
  \(cov1_k \sim \text{Binomial}(p=0.45)\).
\item
  Draw \texttt{cov2} from a binomial distribution:\\
  \(cov2_k \sim \text{Binomial}(p=0.15)\).
\item
  Draw \texttt{cov3} from a binomial distribution:\\
  \(cov3_k \sim \tex{Binomial}(p=0.30)\).
\item
  Draw age at entry from a truncated gamma distribution with shape 0.9
  and rate 0.15, constrained between 60 and 96:\\
  \(A_k \sim \text{Gamma}(\alpha = 0.9, \beta = 0.15), \quad 60 \leq A_k \leq 96\).

  \hypertarget{mm-patterns-chronic-diseases-and-survival}{%
  \subsubsection{\texorpdfstring{\textbf{\emph{MM patterns, chronic
  diseases and
  survival}}}{MM patterns, chronic diseases and survival}}\label{mm-patterns-chronic-diseases-and-survival}}
\item
  Simulate cluster at entry in the study and then simulate the diseases
  at baseline conditioned on the clusters to which they belong. For each
  disease, draw from a binomial distribution with probability \(p\) from
  the latent class model (i.e., the probability of developing a certain
  disease given a multimorbidity cluster and the age at entry).
\item
  Simulate latent multimorbidity cluster trajectories using a multistate
  model with a Gompertz hazard, adjusted for the three binary
  covariates.
\item
  Simulate the prevalent diseases (among those the patient has not yet
  developed), conditioned on the latent cluster towards which the
  patient is transitioning. If the next state of transition is Death,
  then diseases are simulated based on the current state.
\item
  Simulate the age of onset for each developed disease from a truncated
  beta distribution based on disease-specific parameters. The
  distribution is truncated so that the age of onset falls between the
  transition from the previous state to the following. If the computed
  age exceeds the age of death, the corresponding simulated disease is
  discarded.
\item
  Simulate rare diseases independently of the multimorbidity pattern to
  which patients belong but dependently on the patient's age. Rare
  diseases are drawn from a binomial distribution with parameter \(p\)
  equal to the prevalence of such diseases stratified by age.
\item
  Remove subjects who do not present multimorbidity at baseline.

  \hypertarget{end-of-the-observation-period}{%
  \subsubsection{\texorpdfstring{\textbf{\emph{End of the observation
  period}}}{End of the observation period}}\label{end-of-the-observation-period}}
\item
  Draw right-censoring time from a uniform distribution:\\
  \(T_k \sim \text{Uniform}(0.5, 20)\).
\item
  Compute the age of exit from the study as the minimum between the age
  of death and the sum of the age at entry and censoring time.
\item
  Eliminate data after the age of exit (in the case of patients who
  leave the study before dying).

  \hypertarget{study-design}{%
  \subsubsection{\texorpdfstring{\textbf{\emph{Study
  design}}}{Study design}}\label{study-design}}
\end{enumerate}

Based on the data generation mechanism described, the underlying
``true'' datasets (ground truth) are obtained. Then, study design
schemes are applied to derive the observed data through the study.

Following a specific observation schema, visit times are recorded for
each subject, and the exact onset times of diseases are replaced by
binary variables indicating whether a diagnosis was observed at the
visit.

For schema A and schema B, visit times are deterministically simulated
starting from the age of entry into the study.

In contrast, for schema C, visit times are simulated using a Weibull
distribution with parameters shape = 5 and scale = 0.4, starting from
the age of entry.

\hypertarget{examples}{%
\subsection{\texorpdfstring{\textbf{\emph{Examples:}}}{Examples:}}\label{examples}}

\begin{itemize}
\item
  \textbf{Subject A}: Diabetes onset at age 60, study entry at age 65.
  The diabetes diagnosis variable is set to 1 at the baseline visit
  (first visit in the dataset).
\item
  \textbf{Subject B}: Diabetes onset at age 65, study entry at age 62.
  The diabetes diagnosis variable is set to 1 at the closest visit after
  age 65 in a scenario without under-reporting. In a scenario with
  under-reporting, the diabetes diagnosis might appear later if an
  under-reported case is simulated for Subject B at the closest visit.
\end{itemize}

\end{document}
